% !TEX root = ../../Dissertation.tex

\begin{refsection}


\chapter{Conclusions and Outlook}\label{conclusion}

\section{Conclusions}

This dissertation presents the main results achieved during my PhD research. Briefly, the contents of this dissertation focus on studies of small clusters containing 3\textit{d}-block metals. The dissertation stream can be divided into two sequential parts: (i) the study of small 3\textit{d}-metal doped clusters by using highly accurate levels of theory, and (ii) the identification of geometric, electronic structures, and related properties of larger clusters containing more than one transition metal atom. In the first part, clusters were studied in the increasing order of multireference characteristics by raising the number of electrons one by one in the 3\textit{d} subshell of the doping metals. Additionally, several \acrshort{dft} functionals were also employed to assist in the determination of stable geometrical structures, ground and excited states. Furthermore, the use of various density functionals provided basic assessment on how well density functionals perform for systems with increasing magnitude of multireference features. As an inherent shortcoming of multiconfigurational approaches, \acrshort{casscf}-based methods cannot be used for larger systems (larger active spaces needed), and therefore density functional theory remains the choice at the time being. Some density functionals, screened from the first part of this dissertation, were tentatively utilized for next larger clusters chemically composed of multiple transition metal atoms. However, utilization of  more than a half-dozen functionals (to ensure accurate results) for large clusters is still time consuming and requires a large amount of computing resources. Hence, another purpose in the second part of this dissertation is to confine the range of functionals that can handle strong multireference systems to less than a few for the study of future related systems.       



Chapter 2 gave a brief outline of the methodologies employed in this work. In Chapter \ref{ScSi2}, geometry, electronic ground and excited states on the energetic hypersurface of the experimentally known \ch{ScSi2^{-/0}} were investigated and reported. Scandium has one electron in its 3\textit{d} subshell, and therefore, the electronic structures of \ch{ScSi2^{-/0}} are characterized with single-configurational features as expected. Indeed, all dominant electronic configurations of ground and excited states have contributions of $\geq$ 80\% to the wave functions. All tested density functionals can consistently locate ground states of the neutral and anion in accordance with the \acrshort{caspt2} method. All experimental detachment energies obtained from the experimental spectra were also produced correctly as well at several single-reference levels (\acrshort{dft} and \acrshort{rccsd}(T)), and confirmed at the \acrshort{caspt2} level. The point here is that single-reference methods cannot be used to reliably probe low-lying states represented by hyper open-shell electronic configurations \cite{con:1} ($n \geq 2S + 2$ where $n$ is the number of unpaired electrons and $S$ is the total electronic spin). And for the anionic cluster \ch{ScSi2-}, the hyper open-shell configuration of the state \ch{^1B2} (\ch{C_{2v}}) is energetically competitive with the ordinary open-shell configuration of \ch{^3B2}. In this case, \acrshort{dft} methods are not the effective choices for treatment of such electronic degeneracy.   




Titanium digermanium and its anionic form were studied and reported in Chapter \ref{TiGe2}. Because there are more electrons in the 3\textit{d} subshell and the moiety \ch{Ge2} is considered electronically more complex than the \ch{Si2} moiety, electronic structures of \ch{TiGe2^{-/0}} were found to be more complicated than those of \ch{ScSi2^{-/0}}. Leading \acrshort{hf} configurations of few wave functions of \ch{TiGe2^{-/0}} are found to have $\geq$ 80\% contribution, and several leading configurations of other wave functions possess contributions in a range of 40 -- 76\%. With different magnitudes of leading configurations, especially for some states with low magnitudes, the multiconfigurational methods used in this chapter (\acrshort{casscf}/\acrshort{caspt2} and \acrshort{casscf}/\acrshort{nevpt2}) and the golden standard method \acrshort{ccsd}(T) simultaneously gave the same ground states of the neutral and anionic \ch{TiGe2^{-/0}}. Other nearly degenerate states and low-lying states were also identified at these levels. Multiple electronic transitions were proven to cause visible anion photoelectron bands in the experiment. Due to the wide range of leading coefficients, several states of \ch{TiGe2^{-/0}} are believed to be challenging to \acrshort{dft} methods. By comparing electronic energies calculated using several \acrshort{dft} functionals, we found that the M06-L functional can give the same energetic ordering of several electronic states as \acrshort{caspt2}, and \acrshort{nevpt2}, and \acrshort{rccsd}(T) did. 



In Chapter \ref{VGe3}, multiconfigurational characters of \ch{VGe3^{-/0}} were found to be relatively stronger than those of \ch{TiGe2^{-/0}}. The \acrshort{caspt2} method determined that \ch{VGe3-} in the ground state ($^1$A$_1$) has a tetrahedral form with a spatial symmetry of C$_{3v}$. All electronic transitions forming the ground and excited states of the neutral cluster start from the anionic ground state $^1$A$_1$. Upon removal of one frontier electron from the ground state, the resulting neutral cluster undergoes a Jahn-Teller effect, and there are two nearly degenerate electronic states generated. This means that the first band in the experimental spectra is induced by two electronic transitions from the anionic ground state. The same effect is also believed to contribute to the second band. In dealing with multiconfigurational features, the BP86 functional is found to be appropriate for the treatment of \ch{VGe3^{-/0}}. 



To raise the level of multiconfigurational features, in Chapter \ref{Cr2O2} clusters containing two chromium atoms were thoroughly studied. Leading electron configurations of most electronic states have quite small coefficients. These values imply that the \ch{Cr2O2^{-/0}} clusters have very strong multireference features. This is why several tested functionals, except for the pure TPSS functional, cannot produce state energies that are consistent with the \acrshort{raspt2} energies. By using two multireference methods \acrshort{raspt2} and \acrshort{dmrg}-\acrshort{caspt2}, two 10-tet states (\ch{^{10}A_{g}} and \ch{^{10}B_{2g}}) of \ch{Cr2O2-} were identified to be nearly degenerate and proven to be populated simultaneously in the experiment. Two nearly degenerate states were found to cause two first bands in the anion photoelectron spectra. Because of small difference in relative energy, the less-populated anionic ground state \ch{^{10}B_{2g}} causes the very low intensity band X', and the removal of one electron from the true ground state \ch{^{10}A_g} induces the X band with higher intensity in the anion photoelectron spectrum of \ch{Cr2O2-}. 30 electronic transitions (within the active space used) starting from two nearly degenerate states of the anion \ch{Cr2O2-} were predicted to be the origin of removed electrons recorded as several anion photoelectron bands in the spectra.   



We can see that for single-reference clusters like \ch{ScSi2^{-/0}} single-reference methods are good choices. The \acrshort{dft} options for single-reference clusters are quite plentiful. For systems with stronger multireference characters, we found that there can be appropriate functionals for studying these systems. Therefore, we believe that for larger systems containing multiple metallic atoms there will be also good density functionals for investigating these systems. In the next chapter (Chapter \ref{CrxOy}), several functionals were, thereby, used to identify geometric and electronic structures of chromium oxide clusters.





Various unsaturated chromium oxide clusters (\ch{Cr_mO_n+}, m = 2 -- 4, n $\leq$ m) were synthesized, mass-selected, and probed by using the \acrshort{irmpd} technique. Because these clusters are unsaturated and composed of multiple chromium atoms, they are believed to be strongly multiconfigurational. As can be seen above, the TPSS functional can correctly determine the ground and other low-lying states of \ch{Cr2O2^{-/0}}. Therefore, in this work, the TPSS functional was used to study geometric and electronic structures of \ch{Cr_mO_n+}. However, only the results calculated from the TPSS functional are not reliable enough to ascertain the most stable isomers and ground spin states. To increase the reliability of \acrshort{dft} results, six other density functionals were also utilized to probe these chromium clusters. The seven functionals (B3LYP, B3P86, BP86, BPW91, B3PW91, TPSS, TPSSh) used gave consistent results. All experimentally populated chromium oxide clusters were identified by comparing the simulated \acrshort{ir} spectra with the experimental ones. From electronic structures of experimentally populated states, magnetic behaviors and magnetic interaction between magnetic sites within each cluster were revealed. Two main factors control total magnetic moments of these chromium clusters are 3\textit{d}-3\textit{d} bonding-like formation and 2\textit{p}-3\textit{d} delocalization. Magnetic evolution of the series \ch{Cr3O_n+} (n = 0 -- 5) under the addition of oxygen atoms was drawn. 3\textit{d}-3\textit{d} bonding-like formation between two closer chromium atoms is dominant, leading to low total magnetic moments when the quantity of chromium atoms in the clusters dominates that of oxygen atoms. If more oxygen atoms are added, 2\textit{p}-3\textit{d} delocalization becomes dominant, and as a result, total magnetic moments are reduced. 



From the previous chapter, three functionals (TPSS, B3P86, and BP86) were found to be good and therefore selected for IR simulations of clusters containing multiple 3\textit{d}-block metallic atoms. In Chapter \ref{CrMnO}, various bimetallic Cr-Mn oxide clusters (\ch{Cr_xMn_yO_z+}, x + y = 2 -- 4, z = 4 -- 9) were synthesized and characterized with the \acrshort{irmpd} spectroscopy. Bimetallic oxide clusters with different ratios between chromium and manganese were controlled through mass selection. Geometric and electronic structures of all clusters with well-resolved \acrshort{irmpd} spectra were identified in combination with the simulated spectra. We see that the Cr-Mn ratios and addition of oxygen atoms do not affect geometries and general motifs of all hybrid oxide clusters. Two general trends of magnetic properties could be seen from all series of Cr-Mn oxide clusters. Total magnetic moments of Cr-Mn oxide clusters are usually lower than those of corresponding pure chromium oxide clusters. In all hybrid clusters, averaged absolute local magnetic moments of manganese sites are larger than those of chromium sites because of the electron transfer from chromium to manganese sites. Further, electrons captured by additional oxygen atoms are believed to cause reduction of total magnetic moments.  




Overall, this dissertation comprises a series of studies, starting from small clusters with different levels of multiconfigurational features to larger clusters containing multiple metallic atoms. A summary of all studied clusters and general results in this dissertation are tabulated in Table \ref{tblc:summary}. For single-reference systems and study of ground states, single-reference methods such as \acrshort{dft} and \acrshort{ccsd}(T) are good enough for the study of geometry and electronic structures, and some other related properties. But for studying excited states and other electronic states characterized with multireference features, multiconfigurational methods are required; even optimization of geometry needs to be done at multiconfigurational levels. In dealing with larger systems, multiconfigurational methods are practically impossible, and \acrshort{dft} is the only choice. So, in this dissertation, we found that there are some specific relevant functionals for study of multiconfigurational systems.   


\begin{table}[htb!]
    \centering
    \begin{threeparttable}
    \caption{Summary of all studied systems and general results}
    \label{tblc:summary}
    \begin{tabular}{@{}llll@{}}
    \toprule
    system    & calculation method       & experimental data      & result\tnote{($\alpha$)}  \\ \midrule
\ch{ScSi2^{-/0}}  & \begin{tabular}[c]{@{}l@{}}\acrshort{casscf}/\acrshort{caspt2},\\ \acrshort{mrci}(Q), B3LYP,\\ BP86, \acrshort{rccsd}(T)\end{tabular}  & Anion \acrshort{pe} spectra & \begin{tabular}[c]{@{}l@{}} anion: C$_{2v}$, $^3$B$_2$   \\ neutral: C$_{2v}$, $^2$B$_2$        \\ 6, B3LYP, BP86 \end{tabular} \\ \midrule
\ch{TiGe2^{-/0}}  & \begin{tabular}[c]{@{}l@{}}\acrshort{casscf}/\acrshort{caspt2},\\ \acrshort{nevpt2}, M06L,  \\ \acrshort{rccsd}(T)\end{tabular}        & Anion \acrshort{pe} spectra & \begin{tabular}[c]{@{}l@{}} anion: C$_{2v}$, $^4$B$_1$   \\ neutral: C$_{2v}$, $^3$B$_1$        \\ 9, M06L \end{tabular}        \\ \midrule
\ch{VGe3^{-/0}}   & \begin{tabular}[c]{@{}l@{}}\acrshort{casscf}/\acrshort{caspt2},\\ \acrshort{mrci}(Q), BP86, \\  \acrshort{rccsd}(T)\end{tabular}       & Anion \acrshort{pe} spectra & \begin{tabular}[c]{@{}l@{}} anion: C$_{3v}$, $^1$A$_1$   \\ neutral: C$_s$, $^2$A$'$, $^2$A$''$ \\ 6,  BP86 \end{tabular}       \\ \midrule
\ch{Cr2O2^{-/0}}  & \begin{tabular}[c]{@{}l@{}}\acrshort{rasscf}/\acrshort{raspt2},\\ \acrshort{dmrg}-\acrshort{caspt2},   \\  TPSS, BP86\end{tabular}     & Anion \acrshort{pe} spectra & \begin{tabular}[c]{@{}l@{}} anion: D$_{2h}$, $^{10}$A$_g$\\ neutral: D$_{2h}$, $^9$B$_{2g}$     \\  30, TPSS \end{tabular}      \\ \midrule
\begin{tabular}[c]{@{}l@{}} \ch{Cr_mO_n+}     \\ m = 2 -- 4     \\ n $\leq$ m  \end{tabular} & \begin{tabular}[c]{@{}l@{}}B3LYP, B3P86,\\ B3PW91, BP86,\\  TPSS, TPSSh,\\  M06L\end{tabular} & \acrshort{irmpd} spectra     & \begin{tabular}[c]{@{}l@{}}Most stable isomers,\\ electronic ground states,\\ magnetic properties\end{tabular}         \\  \midrule
\begin{tabular}[c]{@{}l@{}} \ch{Cr_xMn_yO_z+} \\ x + y = 2 -- 4 \\ z = 4 -- 9 \end{tabular} & TPSS, BP86, B3P86  & \acrshort{irmpd} spectra   & \begin{tabular}[c]{@{}l@{}}Most stable isomers,\\ electronic ground states,\\ magnetic properties\end{tabular} \\ \bottomrule
    \end{tabular}
    \begin{tablenotes}
        \item[($\alpha$)] For the first 4 systems, the results include symmetry and electronic states of anionic and neutral clusters, numbers of predicted electronic transitions, and good density functionals used.
    \end{tablenotes}
    \end{threeparttable}
\end{table}

%\FloatBarrier


\section{Outlook} 


It is clear that if there are more electrons in the 3\textit{d}-subshell of 3\textit{d}-block metals, study of systems doped with these metals is more challenging. In this dissertation, clusters containing transition metals up to manganese were considered. For small cluster systems containing one and two metallic atoms, we can see dozens of experimental works on synthesis and spectroscopic measurements. \cite{con:2, con:3, con:5, con:6, con:7, con:8, con:9, con:10, con:11, con:12} Similar to small clusters studied in the first part of this dissertation, these clusters are still not fully studied, and therefore, using highly accurate quantum chemical methods to study these systems is of interest. With recent improvements in the development of new methods such as \acrshort{dmrg}-\acrshort{caspt2} \cite{con:dmrgcaspt2} and \acrshort{dmrg}-\acrshort{nevpt2} \cite{con:dmrgnevpt2:1, con:dmrgnevpt2:2}, a large active space consisting of up to 50 orbitals is possible, and the size of clusters which can be studied by highly accurate methods is increased over the limit of \acrshort{rasscf}/\acrshort{raspt2}. This might open a new door for us to use these methods (\acrshort{dmrg}-\acrshort{caspt2} and \acrshort{dmrg}-\acrshort{nevpt2}) to study clusters consisting of three or more transition metal atoms. 


Of the several experimentally synthesized and spectroscopically probed species mentioned above, clusters containing chromium, manganese, and iron are of high priority. To be more detailed, ground and excited states of \ch{Cr2O_n^{-/0}} (n = 2 -- 6) were experimentally reported \cite{con:9} specifying that several ionization processes cause well-resolved bands in the anion photoelectron spectra. With the study on \ch{Cr2O2^{-/0}}, \cite{con:cr2o2} we believe that other \ch{Cr2O_n^{-/0}} clusters are also featured with strong multireference wave functions. To study these systems, multiconfigurational methods are recommended. The series of transition metal oxide clusters which would be in the list of our future works is \ch{Fe2O_n^{-/0}}. Visually, anion photoelectron spectra of \ch{Fe2O_n-} seem to be simpler than those of \ch{Cr2O2-}, \cite{con:8, con:9} but some basic calculations have determined that this series of oxide clusters have even stronger multiconfigurational features than \ch{Cr2O2^{-/0}}. For such inherent multireference systems, we hope that some good density functionals can be identified for the study of \ch{Fe2O_n^{-/0}}. 



Starting from multiconfigurational results, suitable density functionals are expected to be found for the application to larger systems. This is what we have done so far for the cases of \ch{Cr_mO_n+} and \ch{Cr_xMn_yO_z+} clusters. We would expect that iron oxide, manganese oxide, and cobalt oxide clusters, and their properties are our next consideration. In addition to these single-metal oxide clusters, some other hybrid metal oxide clusters containing two or more transition metals (Cr, Mn, Fe, and Co) are also interesting. On the basis of our experience in the study of \ch{Cr_xMn_yO_z+} clusters, we can see that study of clusters containing multiple transition metals is an enormous challenge. And therefore, more time, computing resources, and experimental techniques are required.   



As we can observe that multireference methods cannot be employed for treating large clusters consisting of multiple metallic atoms, and density functional theory is the only current solution. As expected, \acrshort{dft} methods cannot give highly accurate results, and different functionals can lead to different geometrical structures and electronic ground states. Efforts have been made by introducing multiconfigurational pair-density functional theory \cite{con:pairDFT} to overcome the shortcomings of \acrshort{caspt2} and \acrshort{raspt2} in terms of computing resource and analytic gradients. \cite{con:pair-gra} However, it is not enough for treatment of large clusters due to this new technique only recovers dynamic correlation on the basis of \acrshort{casscf} and \acrshort{rasscf} wave functions. Therefore, a new \acrshort{casscf} and \acrshort{rasscf} algorithm which can handle a larger number of orbitals in the active space is in demand. A newly developed method, the \acrshort{dmrg}-\acrshort{caspt2}, can possibly treat larger system with up to 50 orbitals in the active space as mentioned above, but this method requires a huge amount of computing resource and time. Solutions for analytical energy gradients at the \acrshort{dmrg}-\acrshort{caspt2} level are still not available, which blocks this new method from application. Recently, the \acrshort{dmrg}-SCF with analytical gradients has been developed, \cite{con:dmrgscf-gra} which means this technique can be used for geometrical optimization more efficiently but still the dynamic correlation needs to be incorporated appropriately. A question can, now, be proposed whether dynamic correlation energy recovered from perturbative calculations is important if a larger number of orbitals in the active space is taken into account at the \acrshort{dmrg}-SCF step. If it is not important (for specific sizes of systems), the \acrshort{dmrg}-SCF with analytic gradients can be applied to large clusters for geometrical optimization without much loss of accuracy and any worry about the bottleneck caused by \acrshort{caspt2} and \acrshort{nevpt2}.
 

%- When a large number of orbitals are treated in the active space, second order perturbation is still needed because this step need lots of computing resource and time for both CASSCF and DMRG wave functions.
 




%%%%%%%%%%%%%%%%%%%%%%%%%%%%%%%%%%%%%%%%%%%%%%%%%%
% Keep the following \cleardoublepage at the end of this file, 
% otherwise \includeonly includes empty pages.
%\cleardoublepage

\includebibliography
\printbibliography[heading=subbibliography] % print section bibliography




\end{refsection}
