% !TEX root = ../../Dissertation.tex

\chapter{Abstract}                                 \label{ch:abstract}


This dissertation is dedicated to study of small clusters containing 3\textit{d}-block metals. Although these clusters can be synthesized and probed by experimental techniques, they are geometrically and electronically unknown. Data on clusters obtained from the experiment supply initial  information about synthesized clusters like stoichiometry, and such primary information can be used as very reliable reference for identification of atomic arrangements and electronic structures of clusters. Additional secondary properties of the studied clusters can also be drawn once electronic and geometric structures of clusters are determined. To identify geometry of atomic combination and electronic structures, quantum chemical methods such as wave function based approaches and density functional theory can be used for reproduction of and interpretation on experimental outcomes. In this doctoral study, two categories of experimental data measured are ionization energy or detachment energy of electrons removed from clusters under irradiation of photon laser beams, and the unique fingerprint vibrations of atomic components within clusters. Detachment energies of electrons removed from anionic clusters are measured and recorded in anion photoelectron spectra. The vibrations of atoms in this dissertation are mainly measured by using the infrared multiphoton dissociation spectroscopic technique; and therefore, obtained vibrations are recorded in so-called infrared multiphoton dissociation spectra. 


The first part of this thesis including Chapter \ref{ScSi2} to Chapter \ref{Cr2O2} focuses on the small clusters that have been studied with highly accurate methods (\acrshort{casscf}/\acrshort{caspt2}, \acrshort{nevpt2}, \acrshort{mrci}, \acrshort{dmrg}-\acrshort{caspt2}, and \acrshort{ccsd}(T)). By applying these methods and comparing calculated results with experimental ones, geometries, ground and excited states of the studied clusters can be determined correctly. From the electronic structures and calculated detachment energies of outer electrons, all possible electronic transitions within a limit of theoretical methods can be also predicted and all observed experimental bands (corresponding to specific detachment processes) can be understood and assigned. To further support band assignments made on the basis of theoretical calculations, we used simulations of detachment bands by calculating multidimensional Franck-Condon factors. The simulated bands can provide more detailed insights into each experimental band. In parallel with highly accurate methods mentioned above, \acrshort{dft}-based methods were used thoroughly for all studied systems. 




It is worth noting that all clusters studied in each chapter from Chapter \ref{ScSi2} to Chapter \ref{Cr2O2} were arranged in an order of increasing multireference features indicated by the coefficients of dominant configurations. The more the studied systems are multiconfigurational, the more they are challenging to single reference methods. With the increasing multiconfigurational features and the testing use of \acrshort{dft} methods, we would like to find out suitable density functionals for study of similar systems in terms of magnitude of multireference. Indeed, for single reference systems like \ch{ScSi2^{-/0}}, all tested functionals and the \acrshort{rccsd}(T) method could well reproduce the experimental ionization energies of all electronic transitions in comparison to values obtained from the experiment and multireference methods. More importantly, we always find out relevant functionals which can relatively reproduce the experimental results, even though the system studied has strong multireference features like \ch{Cr2O2^{-/0}}. Overall, there are no general rules for selection of functionals to deal with specific levels of multireference. Such a selection becomes an empirical exercise. 




The second part (Chapters \ref{CrxOy} and \ref{CrMnO}) of this dissertation concentrates on a bit larger clusters containing multiple transition metal atoms. To be more specific, all clusters studied in this part are transition metal oxides in the gas phase (\ch{M_mO_n+}). Owing to possession of multiple transition metal atoms, correct wave functions of clusters studied in this part are believed to be multiconfigurational. But due to the large sizes of these clusters, they cannot be studied with high-accuracy methods mentioned above. The only possible choice here is to use density functional theory barely. As we have no general rules for choosing good functionals, a set of several functionals collected from all studies in the first part and some additional popular functionals were utilized to study \ch{Cr_mO_n+} clusters. Geometric arrangements of atoms were discovered by comparing the simulated IR spectra with the experimental ones. Magnetic interplay, evolution, and insights of these chromium oxide clusters were revealed. For the mixed \ch{Cr_xMn_yO_z+} clusters, three functionals selected from the previous study were used for geometric identification. Geometrical and magnetic evolution of these clusters was also studied following addition of oxygen atoms and by changing the ratio of the different metals. 






%%%%%%%%%%%%%%%%%%%%%%%%%%%%%%%%%%%%%%%%%%%%%%%%%%
% Keep the following \cleardoublepage at the end of this file, 
% otherwise \includeonly includes empty pages.
\cleardoublepage

% vim: tw=70 nocindent expandtab foldmethod=marker foldmarker={{{}{,}{}}}
