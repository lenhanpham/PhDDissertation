% !TEX root = ../../Dissertation.tex

\begin{refsection}

\chapter{Introduction}\label{ch:introduction}


%%% General ideas about transition metals and clusters containing transition metals: Features, Properties, Applications

3\textit{d}-block metals are elements whose electronic structures are characterized by the ongoing filling-up of the 3\textit{d} orbitals along the series. In the Periodic Table, 3\textit{d}-block metals,  except for scandium and zinc, are known as the first-row transition metals, bridging two sides of \textit{s}- and \textit{p}-block elements. As mentioned above, five 3\textit{d} orbitals of the 3\textit{d}-block atoms are gradually filled with electrons in order of increasing atomic number. Due to these 3\textit{d} electrons, transition metals expose several different properties in comparison to main group elements, for example, multiple oxidation states and magnetic field manifestation found in bulk iron, cobalt, and nickel. Different oxidation states of transition metals can readily interconvert owing to incomplete 3\textit{d}-filled orbitals, which facilitates catalytic capability of transition metals and their complexes. \cite{c1:1, c1:2, c1:3, c1:4} Interestingly, in forms of clusters, transition metals can chemically and physically behave in such a way that they are totally unique and unexpected when compared to corresponding bulk materials. With unique and unexpected properties of transition metals, clusters of materials doped with or containing transition metals are also expected to have new features and/or their existing properties will be tuned. Adding one or more 3\textit{d}-block metallic atoms to pristine clusters is a popular procedure that is experimentally and theoretically employed to investigate novel materials. 


%- Talk about experimental methods used to study clusters containing transition metals appearing in this thesis.

Several methods can be used to synthesize and characterize clusters that contain transition metals. In this work, clusters were generated (partially by our experimental coworkers) on the basis of laser vaporization processes, and then probed by two different spectroscopic techniques. The first technique is called anion photoelectron spectroscopy. Anionic clusters generated and mass-selected are irradiated with high-energy photon beams. Under such high-energy photon beams, electrons in anionic clusters are detached, and their kinetic energies are, in turn, measured. All signals obtained under the form of ionization energy are recorded in anion photoelectron spectra. Through ionization energies, one can indirectly identify geometric and electronic structures of synthesized clusters in combination with relevant quantum chemical theories. The second technique is infrared multiphoton dissociation spectroscopy. In this technique, energy of photons which are resonantly absorbed by clusters is recorded by tracing any dissociation happening in the clusters. \cite{c1:5} This technique is also called action spectroscopy and was designed to circumvent all difficulties that prevented applying direct IR spectroscopy to gas phase clusters. Vibrational spectra obtained from this technique provide vibrational details that are directly related to geometric structures of the clusters studied. Again, quantum chemical calculations are used to understand properties of the clusters. 


%- Theoretical methods 

As can be understood from above, quantum chemical calculations are essential to fully understand synthesized clusters. Due to the incomplete 3\textit{d}-filled orbitals, quantum chemical methods employed must be able to recover enough correlation energy to describe clusters containing transition metals. Conveniently, the correlation energy can be divided into two types: static and dynamic correlation energy. For clusters doped with transition metals, electronic structures are not always correctly described by single-reference wave functions. In most cases, multiconfigurational wave functions methods are needed to treat these systems. Multiconfigurational wave functions can recover static correlation energy which cannot be obtained by using single-reference wave functions. Dynamic correlation energy, understood as instantaneous interaction among electrons, can be considered by many popular methods such as \acrshort{dft}, coupled clusters, and perturbation theory. A method that can recover enough both static and dynamic correlation energy can give reliable energy of the studied systems. A popular method is complete active space self-consistent field in combination with second order perturbation theory (\acrshort{casscf}/\acrshort{caspt2}). However, multireference methods like \acrshort{casscf}/\acrshort{caspt2} and \acrshort{mrci} cannot be used for large systems. 



%+ How challenging clusters containing transition metals to theoretical study, why? 
%+ Introduce correlation energies: static and dynamic 
%+ Methods like DFT, CCSD(T), multiconfigurational 
%- Structures of this thesis: Purpose and outline of this thesis

Multireference methods, that can provide reliable results on clusters containing transition metals, can only treat systems with a limited number of orbitals in the active space. This makes multiconfigurational methods applicable to clusters containing one or two transition metal atoms. To study larger clusters containing more transition metal atoms, the possible option is to use \acrshort{dft} methods. The point for using \acrshort{dft} methods to treat systems containing transition metals is about to find appropriate functionals. Therefore, in this work, we used multiconfigurational methods to investigate energetic and electronic properties of small clusters containing one and two 3\textit{d}-block metal atoms. Simultaneously, numerous density functionals were also employed to study these clusters. The results obtained from \acrshort{dft} methods were compared to those obtained from experiments and multireference methods. Good functionals were then selected for larger clusters for which multireference methods cannot be used. To identify good functionals, the complexity of studied systems was increased by choosing clusters doped with more complex metallic atoms (more electrons in the 3\textit{d} subshell). Because large clusters with more transition metal atoms are believed to have strong multireference features, increase of the complexity of small systems allows us to determine suitable functionals that can handle strong multireference systems. 


Briefly, this dissertation can be divided into two main parts: i) study of ground and excited states of small clusters containing one and two 3\textit{d}-block metallic atoms in order of increasing multiconfigurational features and ii) probe of larger clusters with more than two 3\textit{d}-block metallic atoms. All results are included in six out of the nine chapters (Chapter \ref{ScSi2} to Chapter \ref{CrMnO}). Results of each chapter are summarized concisely in the following paragraphs. 


In Chapter \ref{ScSi2}, scandium, the simplest 3\textit{d}-block metal in the Periodic Table, with 1 electron in the 3\textit{d} subshell, is the first object. This metal can form clusters with different main group elements. \cite{c1:6, c1:7, c1:8, c1:9} Various scandium doped clusters have been synthesized and studied. Among them, \ch{ScSi2^{-/0}} is the simplest one synthesized and experimentally probed with anion photoelectron spectroscopy. Experimental data on ionization energies of the anionic cluster \ch{ScSi2-} were precisely measured and reported. Various theoretical studies using single-reference methods (\acrshort{dft}) gave inconsistent conclusions on its ground and excited states. Therefore, to shed light on this cluster, multireference methods are definitely needed. Geometrical structures of several possible low-lying states of both neutral and anion were optimized using the \acrshort{caspt2} method. In addition, the \acrshort{mrci} method was also utilized to confirm the ground and low-lying states on the potential surface. Several single-reference methods, such as \acrshort{rccsd}(T) and \acrshort{dft}, were also used to reinforce our obtained results and test how \acrshort{dft} methods work with clusters containing the simple 3\textit{d}-block metal Sc. 




The next 3\textit{d}-block transition metal in the Periodic Table is titanium. This transition metal has two electrons in its 3\textit{d} orbitals, which is one electron more than Sc in the 3\textit{d} subshell; therefore, clusters containing titanium are believed to be more complex in terms of multireference features. Similar to scandium, several types of clusters containing this transition metal were synthesized and characterized. \cite{c1:10, c1:11, c1:12, c1:13, c1:14, c1:15} In Chapter \ref{TiGe2}, the titanium digermanium clusters \ch{TiGe2^{-/0}} were selected as the next object of this work. There are also several reports that used \acrshort{dft} methods to identify geometry and electronic structures of \ch{TiGe2^{-/0}}. Two multireference methods \acrshort{caspt2} and \acrshort{nevpt2} were used to ensure reliable results on ground and low-lying states. This type of systems was found to be more multiconfigurational than \ch{ScSi2^{-/0}}, and therefore, it is worth employing single-reference methods to see how \acrshort{dft} functionals are able to reproduce experimental ionization energy. All results calculated with single-reference methods were also used to assist the assignment of all ionization processes observed in the experiments.





In the next chapter (Chapter \ref{VGe3}), a species of clusters containing vanadium was selected to evaluate if \acrshort{dft} methods can be used to study stronger multireference systems. From experimentally available clusters containing vanadium, the \ch{VGe3^{-/0}} clusters are the smallest experimentally reported ones. A quick review over previous computational works shows that the use of \acrshort{dft} methods led to different geometry and electronic ground states. No clear-cut conclusions on the most stable geometry, electronic structures, and energetic ionization processes were withdrawn by using only single-reference methods. Therefore, a better way to proceed with the \ch{VGe3^{-/0}} clusters is to use multireference methods. Indeed, the \acrshort{caspt2} method was used to optimize geometry of the clusters, and relative energies of several electronic states were also calculated at this level. All results from \acrshort{caspt2} calculations were then used as reliable references for evaluation of \acrshort{dft} results. In this work, several \acrshort{dft} functionals were employed to calculate relative energies and compared with results from \acrshort{caspt2} calculations. Besides the explanation for all experimental observations, specific density functionals were also found to be good for treatment of systems with more multiconfigurational characters.     




The following transition metal considered in this dissertation is chromium. The \ch{Cr2O2^{-/0}} clusters are the central object in Chapter \ref{Cr2O2}. The complexity of the studied system is increased not only by adding more electrons into the 3\textit{d} subshell but also by including more metallic atoms in the cluster. Two groups independently synthesized and probed clusters of \ch{Cr2O2^{-/0}} with the anion photoelectron spectroscopy technique. \cite{c1:16, c1:17} As expected, the dichromium dioxide clusters \ch{Cr2O2^{-/0}} are very computationally challenging due to strong multiconfigurational insights of this kind of clusters. For studying \ch{Cr2O2^{-/0}}, \acrshort{casscf}/\acrshort{caspt2} cannot be employed because the number of orbitals in the active space is too large for \acrshort{casscf} calculations. Hence, the geometry of \ch{Cr2O2^{-/0}} was optimized at the \acrshort{raspt2} level of theory. The \ch{Cr2O2-} cluster was determined to have close-lying states competing for the ground state. The DMRG-\acrshort{caspt2} method with a larger active space was employed to clarify the correct ground state and other low-lying states. \acrshort{dft} functionals were also used to locate energetic positions of several close-lying states. Similar to other clusters mentioned above, specific density functionals are expected to give a good description to such strong multireference systems like \ch{Cr2O2^{-/0}}. 



From all of the above results, we can observe that there are always suitable functionals for the treatment of strong multireference systems. Therefore, \acrshort{dft} methods can be used to study clusters containing more transition metal atoms with strong multireference characters. The conclusion here is that we do not know which functionals can correctly describe geometrical and energetic properties of these clusters. However, on the basis of previous studies, some popular functionals were empirically opted for our next studies on larger clusters containing more transition metal atoms. 



In the next chapter (Chapter \ref{CrxOy}), we studied a series of unsaturated chromium oxide clusters \ch{Cr_mO_n+} (m = 2 -- 4, n $\leq$ m). This series of oxide clusters was synthesized and spectroscopically characterized. Infrared multiphoton dissociation spectra measured on the synthesized clusters were then used as reliable references for identification of their geometric and electronic structures. Clearly, we can see that the number of chromium atoms in most unsaturated oxide clusters is larger than 2 except for \ch{Cr2O2+}. Multiconfigurational methods cannot be used for all cluster sizes but \ch{Cr2O2+}. Therefore, in this series of oxide clusters, various prevalent functionals were employed to simulate IR spectra of these clusters, and additionally, \acrshort{rasscf}/\acrshort{raspt2} was used for \ch{Cr2O2+} as an energetic benchmark for several single-reference methods. The experimentally populated isomers and their electronic structures obtained from \acrshort{dft} calculations, in turn, provided details of magnetic interaction between metallic sites within each cluster and magnetic evolution under addition of oxygen atoms. Insights into magnetic interplay and evolution were also revealed. As mentioned previously, seven functionals were used for two main purposes, which are to confirm \acrshort{dft} results and to choose some of good functionals for other strong multireference oxide clusters of transition metals.  




Another series of oxide clusters containing chromium atoms is our final object of this dissertation. The difference of this series from the above chromium oxide clusters is that new synthesized clusters are bimetallic. Aside from chromium, manganese is the second transition metal incorporated into these clusters. The main purpose of Chapter \ref{CrMnO} is to investigate the simultaneous effects of two transition metals on the geometrical structures of chromium-manganese oxide clusters and the roles of metallic types in controlling magnetic properties. To accomplish this, vibrational spectra, that are obtained from infrared multiphoton dissociation spectroscopy, were applied as the basis for \acrshort{dft} geometric assignments of all synthesized clusters. Three good functionals selected from the previous chapter were utilized to simulate IR spectra. From all results obtained, geometrical and magnetic evolution of this series of bimetallic clusters were figured out.  




In Chapter \ref{conclusion}, the conclusions will briefly summarize all results obtained in this dissertation. On the basis of current works, we propose an outlook for our future research. All chapters from \ref{ScSi2} to \ref{CrMnO} are main works of this doctoral study and they provide all important data. For all remaining data, each chapter from \ref{ScSi2} to \ref{CrMnO} has relevant supporting information available online. It should be stressed that this is a study using quantum chemical computations. Experimental results were either taken from literature (\acrshort{pe} spectra) or provided by our experimental coworkers, mainly by the group of Prof. Ewald Janssens, Department of Physics KU Leuven, and Prof. Andrei Kirilyuk, Institute of Molecules and Materials, Radboud University Nijmegen, The Netherlands. 





%%%%%%%%%%%%%%%%%%%%%%%%%%%%%%%%%%%%%%%%%%%%%%%%%%
% Keep the following \cleardoublepage at the end of this file, 
% otherwise \includeonly includes empty pages.
%\cleardoublepage

\includebibliography
\printbibliography[heading=subbibliography] % print section bibliography

\end{refsection}


