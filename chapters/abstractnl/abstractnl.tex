% !TEX root = ../../Dissertation.tex

\chapter{Beknopte samenvatting}

Dit proefschrift is gewijd aan de studie van kleine clusters die 3\textit{d}-block metalen bevatten. Deze clusters kunnen worden gesynthetiseerd en gekarakteriseerd door middel van experimentele technieken, maar hun geometrische en elektronische kenmerken zijn onbekend. Gegevens over de clusters verkregen uit experiment en leveren initiële informatie zoals stoichiometrie, en dergelijke primaire informatie kan als zeer betrouwbare referentie voor identificatie van geometrische en elektronische structuren van clusters worden gebruikt. Andere secundaire eigenschappen van de bestudeerde clusters kunnen ook worden bepaald zodra hun elektronische en geometrische structuren zijn gekend. Om de geometrie van atoomcombinaties en elektronische structuren te identificeren, kunnen kwantumchemische methoden zoals golffunctiegebaseerde methoden en dichtheidsfunctionaaltheorieën worden gebruikt voor de reproductie en interpretatie van experimentele resultaten. 


Twee categorieën van de gemeten experimentele gegevens in dit proefschrift zijn ionisatie-energie of verwijderingsenergie van elektronen uit clusters onder bestraling met laserlicht, en vibratiemodes die een unieke vingerafdruk zijn van atomaire componenten binnen clusters. Verwijderingsenergieën van elektronen die uit anionische clusters worden gemeten en geregistreerd in een anion-fotoelektronspectrum. De vibraties van atomen in dit proefschrift worden voornamelijk gemeten met behulp van de infrarood fotodissociatie spectroscopische techniek, en daarom worden de verkregen trillingen opgenomen in infraroodfotodissociatiespectra.



Het eerste deel van dit proefschrift, met inbegrip van hoofdstuken \ref{ScSi2} t/m \ref{Cr2O2}, richt zich op de kleine clusters die met zeer nauwkeurige methoden kunnen worden bestudeerd (CASSCF/CASPT2, NEVPT2, MRCI, DMRG-CASPT2, en CCSD(T)). Door deze methoden toe te passen en de berekende resultaten te vergelijken met de experimentele, kunnen geometrieën, grond- en aangeslagen toestanden van de bestudeerde clusters correct worden bepaald. Uit de elektronische structuren en berekende verwijderingsenergieën van buitenste elektronen kunnen ook alle mogelijke elektronische overgangen binnen de theoretische grenzen van de methoden worden voorspeld en kunnen alle waargenomen experimentele banden (die overeenkomen met specifieke verwijderingsenergieën) worden begrepen en toegewezen. Om de bandtoewijzing op basis van theoretische berekeningen verder te ondersteunen, gebruiken we simulaties van verwijderingsenergieën door multidimensionale Franck-Condon factoren te berekenen. De gesimuleerde banden kunnen meer gedetailleerd inzicht verschaffen in elke experimentele band. Parallel met de zeer nauwkeurige hierboven vermelde methoden, worden DFT methoden ook uitgebreid gebruikt in alle bestudeerde systemen. 


Het is vermeldenswaard dat alle bestudeerde clusters in elk hoofdstuk van hoofdstuk \ref{ScSi2} tot hoofdstuk \ref{Cr2O2} werden gerangschikt in een volgorde van toenemende multireferentie-eigenschappen, aangegeven door de coëfficiënten van de dominante configuraties. Hoe meer de bestudeerde systemen multiconfigurationeel zijn, hoe meer uitdagend ze zijn voor mono-referentiemethoden. Met de toenemende volgorde van multiconfigurationele kenmerken en het testen van het gebruik van DFT-methoden, willen we graag weten welke dichtheidsfunctionalen geschikt zijn voor het bestuderen van gelijkaardige systemen in termen van de omvang van multireferentie. Inderdaad, voor enkelvoudige referentiesystemen zoals \ch{ScSi2^{-/0}}, kunnen alle geteste functionalen en de coupled-cluster RCCSD(T)-methode, de experimentele ionisatieenergieën van alle elektronische overgangen goed reproduceren in vergelijking met de waarden verkregen uit het experiment en multireferentiemethoden. Belangrijker is dat we altijd relevante functionalen vinden die de experimentele resultaten relatief kunnen weerspiegelen, ook al heeft het bestudeerde systeem sterke multireferentie eigenschappen zoals \ch{Cr2O2^{-/0}}. Er zijn geen algemene regels voor het gebruik van functionalen met specifieke niveaus van multireferentie.




Het tweede deel (hoofdstukken \ref{CrxOy} en \ref{CrMnO}) van dit proefschrift concentreert zich op een paar grotere clusters met meerdere overgangsmetaalatomen. Om specifieker te zijn, zijn alle onderzochte clusters in dit deel overgangsmetaaloxide in de gasfase (\ch{Cr_mO_n+} en \ch{Cr_xMn_yO_z+}). Door het bezit van meerdere transitiemetaalatomen zijn de correcte golffuncties van de in dit deel bestudeerde clusters multiconfigurationeel. Als deze clusters groot zijn kunnen zij niet bestudeerd worden met zeer nauwkeurige methoden zoals hierboven vermeld. De enig mogelijke keuze hier is om gebruik te maken van de dichtheidsfunctionaaltheorie. Omdat we geen algemene regels hebben voor de selectie van goede functionalen, hebben we dus een set van verschillende functionalen verzameld uit alle studies in het eerste deel en ook een aantal extra populaire functionalen werden gebruikt om \ch{M_mO_n+} clusters te bestuderen. Geometrieën van de clusters worden ontdekt door een vergelijking van de gesimuleerde IR-spectra met experiment. Magnetische interactie, evolutie en inzichten van deze chroomoxideclusters worden onthuld. Voor \ch{Cr_xMn_yO_z+} clusters, werden drie geselecteerde functionalen uit de vorige studie gebruikt voor geometrische identificatie. Geometrische en magnetische evolutie van deze clusters werden ook bestudeerd onder toevoeging van zuurstofatomen en wijziging van de metaal-metaal verhoudingen. 










%%%%%%%%%%%%%%%%%%%%%%%%%%%%%%%%%%%%%%%%%%%%%%%%%%
% Keep the following \cleardoublepage at the end of this file, 
% otherwise \includeonly includes empty pages.
\cleardoublepage

% vim: tw=70 nocindent expandtab foldmethod=marker foldmarker={{{}{,}{}}}
